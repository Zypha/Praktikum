\section{Brennweite}
In einem Experiment soll die Brennweite einer Linse bestimmt werden. Hierzu
wird eine Lampe im Abstand $g$ vor die dunne Linse ¨ gestellt. Auf der anderen
Seite der Linse befindet sich im Abstand b ein Schirm, der solange verschoben
wird, bis ein scharfes Bild zu sehen ist. Die Brennweite $f$ der Linse läßt sich dann
durch die Linsengleichung berechnen.

\begin{equation}
    \label{eq:Brennweite}
        \frac{1}{f}=\frac{1}{g} + \frac{1}{b}
\end{equation}


Fur sechs verschiedene Gegenstandsweiten $g$ wurde die Bildweite $b$ bestimmt

\begin{table}
\centering
\label{tab:lösunga}
\begin{tabular}{c c}
    Gegenstandsweite $g [\symup{mm}]$ & Bildweite $b [\symup{mm}]$\\
    \midrule
    60 &285 \\
    80 &142 \\
    100& 117\\
    110& 85 \\
    120& 86 \\
    125& 82 \\
    \bottomrule
\end{tabular}
\end{table}

\subsection{verschiedene Brennweiten}
 Berechnen Sie fur die sechs verschiedenen Kombinationen die Brennweite ¨ f der
dunnen Linse. Verwenden sie die Linsengleichung. Berechnen Sie aus den sechs ¨
Brennweiten den Mittelwert, die Standardabweichung und den Fehler des Mittelwertes.

\paragraph{Ansatz und Lösung} \mbox{} \\
Mithilfe der gegeben Linsengleichung (\ref{eq:Brennweite}) lässt sich das entsprehcende $f$ bestimmen.

 \begin{table}
\centering
\begin{tabular}{c c c}
    Gegenstandsweite $g [\symup{mm}]$ & Bildweite $b [\symup{mm}]$ & Brennweite $f [\symup{mm}]$\\
    \midrule
    60 &285 &49.57\\
    80 &142 &51.17\\
    100& 117&53.92\\
    110& 85 &47.95\\
    120& 86 &50.10\\
    125& 82 &49.52\\
    \bottomrule
    
\end{tabular}
\caption{Brennweite folgend aus Gegenstandsweite und Bildweite}
\label{tab:loesung}
\end{table}

\paragraph{Mittelwert} \mbox{} \\
Anschließend wird der Mittelwert berechnet durch entsprechende Formel.
\begin{equation}
\Delta f = \frac{1}{n} \sum_{i=1}^n f_i 
\end{equation}

Wobei $n$ hier die Anzahl der vorgenommmen Messungen ist. Die restlichen Werte einsetzen ergibt den Mittelwert, welcher lautet
\begin{equation}
\label{eq:delta}
\Delta f=\SI{50.2}{\milli\meter}.
\end{equation}

\paragraph{Standardabweichung} \mbox{} \\
Die Standardabweichung resultiert indirekt aus dem Mittelwert. Hierfür wird folgende Forme bennötigt.

\begin{equation}
\sigma_f = \sqrt{\frac{1}{n-1} \sum_{i=1}^n (f_i - \bar{f})^2}
\end{equation}

Aus (\ref{tab:loesung}), (\ref{eq:delta}) und $n=6$ nimmt man die Werte und erhält folgendes Ergenis.

\begin{equation}
\sigma_f = \SI{2.04}{\milli\meter}
\end{equation}

\paragraph{Fehler des Mittelwertes} \mbox{} \\
\begin{multicols}{2}
    \begin{equation}
    \label{eq:Fehler}
    \increment \bar{f} = \sqrt{\frac{1}{n(n-1)} \sum_{i=1}^n (f_i - \bar{f})^2}
    \end{equation}\break
    \begin{equation}
    \label{eq:Fehlerkurz}
    \increment \bar{f} =\frac{s}{\sqrt{n}}
    \end{equation}
\end{multicols}

Analog zur Stadnartabweichung lassen sich nun die Messwerte und $n=6$ einsetzen wodurch man mit der Formel \eqref{eq:Fehler} oder \eqref{eq:Fehlerkurz}
 den entsprechenden Fehler des Mittelwelwertes bekommt.
\begin{equation}
\increment \bar{f}= \SI{0.83}{\milli\meter}
\end{equation}
\newpage

\subsection{Diagramm}
\paragraph{G-B Diagramm} \mbox{} \\

\begin{figure}
    \centering
    \includegraphics[width=\textwidth]{build/plot2.pdf},
   \caption{G-B Diagramm}
   \label{fig:G-B}
\end{figure}

Die kleinen Werte resultieren aus der inversen Betrachtung des Brennwertes $f$.

\paragraph{Lineare Regression} \mbox{}\\

\begin{equation}
\label{eq:LR}
\hat{f}_k = bx_k +a 
\end{equation}

\begin{align*}
&\hat{f}_k   &&\text{- Vorhergesagter Wert auf dem Kriterium $y$ für den k-ten Messwert} \\
&x_k  \hspace{1cm} &&\text{- k-ter Messwert auf dem Prädiktor x}  \\
&b  &&\text{- Regressionsgewicht, Steigung der Regressionsgeraden } \\
&a &&\text{- $y$ - Achsenabschnitt der Regressionsgeraden} 
\end{align*}

\begin{figure}
    \centering
    \includegraphics[width=\textwidth]{build/plot3.pdf},
   \caption{G-B Diagramm}
   \label{fig:G-BL}
\end{figure}
\newpage
\subsection{Vergleich der Ergebnisse}
Die gegebenen Werte und der Graph der Linearen Regression liegen nahe beeiander und im Berecih der Stadnardabweichung. 