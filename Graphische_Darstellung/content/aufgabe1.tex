\section{Federkonstante}
Die Federkonstante k einer Federwaage soll nach dem Hookschen Gesetz F =
k ·x bestimmt werden. Hierzu werden verschiedene Gewichte m an die Federwaage geh¨angt und die jeweilige Ausdehnung x gemessen.

\paragraph{Messdaten:} \mbox{} \\

\begin{table}
    \centering
    \caption{Gewicht zur zugehöriger Auslenkung}
    \label{tab:Gewicht}
        \begin{tabular}{c c}
        \toprule
        m [g] & x [cm] \\
        \midrule
        2 & 1.6      \\
        3 & 2.7     \\
        4 & 3.2   \\
        5 & 3.5  \\
        6 & 4.0 \\
        \bottomrule
    \end{tabular}
\end{table}

\paragraph{m-x Diagramm} \mbox{} \\

\begin{figure}
    \centering
    \includegraphics[=\textwidth]{../build/plot1.pdf}
    \caption{m-x Diagramm}
    \label{fig:plot1}
\end{figure}