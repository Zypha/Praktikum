\newpage
\section{Absorbationsgesetz}

Als erstes gil tes die Messunsciherheit für $N$ zu berchnen. Dafür bedient man sich dem Absorbationsgesetz

\begin{equation}
    N=N_0 * e^{-\mu d}
\end{equation}

 und der Angabe zur Berechnung der Unischerheit

\begin{equation}
\label{eq:fehler}
   \Delta N= \sqrt{N}
\end{equation}

Diese angewandt auf die Daten ergibt folgende Tablle

\begin{table}
    \centering
    \caption{$d-N$ Diagramm mit Fehler}
    \begin{tabular}{c c c}
        \toprule
        d [cm] & N [1/60s] & Fehler ($=\sqrt{N}$)\\
        \midrule
        0.1 & 7565  & 86.97 \\
        0.2 & 6907  & 83.10 \\
        0.3 & 6214  & 78.82 \\
        0.4 & 5531  & 74.37 \\
        0.5 & 4942  & 70.29 \\
        1.0 & 2652  & 51.49 \\
        1.2 & 2166  & 46.54 \\
        1.5 & 1466  & 38.28 \\
        2.0 & 970  & 31.14 \\
        3.0 & 333  & 18.24 \\
        4.0 & 127  & 11.26 \\
        5.0 & 48  & 6.92 \\
        \bottomrule
    \end{tabular}
\end{table}

\subsection{$lin lin$ Darstellung} 
Die lineare Darstellung auf beiden Achsen des Vorgangs sieht wie folgt aus.
\begin{figure}
    \centering
    \includegraphics[width=\textwidth]{build/plot4.pdf},
   \caption{$lin-lin$ Darstellung der Absorbation - linear}
   \label{fig:linlin}
\end{figure}

Der Fehler wurde analog zur Anweisung mit \eqref{fehler} berechnet und dem Plot hinzugefügt.

\subsection{$lin log$ Darstellung} 
Hier wird jeweils von der kompletten $y-Achse$ der natürliche Logarithmus genommen um eine entsprechende halblogarithmische Darstellung zu erhalten.
Die Dicke wird durch die $x-Achse$ beschrieben und bleibt unverändert.

\begin{figure}
    \centering
    \includegraphics[width=\textwidth]{build/plot5.pdf},
   \caption{$lin-log$ Darstellung der Absorbation -halblogarithmisch}
   \label{fig:linlin}
\end{figure}


\subsection{Absorptionskoeffizienten}
Da man nun die Art und das Aussehen der Werte kennt, kann man sie versuchen mit einer Funktion durch einen $curvefit$ zu beschreiben. 
Dadurch erhält man einen ungefähren Zahlenwert für den Absorptionskoeffizienten. Dieser liegt laut Python bei den $\mu = 1.139$, bzw mit diesem Wert wurde der folgende Graph dargstellt.


\begin{figure}
    \centering
    \includegraphics[width=\textwidth]{build/plot6.pdf},
   \caption{$lin-log$ Darstellung der Absorbation}
   \label{fig:linlin}
\end{figure}

